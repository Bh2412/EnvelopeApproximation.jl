
\documentclass{article}
\usepackage{graphicx} % Required for inserting images
\usepackage[utf8]{inputenc}
\usepackage{graphicx}
\usepackage[toc,page]{appendix}
\usepackage{tikz}
\usepackage{amsmath}
\usetikzlibrary{shapes.geometric, arrows}
\tikzstyle{startstop} = [rectangle, rounded corners, minimum width=3cm, minimum height=1cm,text centered, draw=black, fill=white]
\tikzstyle{arrow} = [thick,->,>=stealth]

\title{stress energy tensor computation notes}
\author{Ben Hatzofe }
\date{March 2024}

\begin{document}

\maketitle

\section{Introduction}
A late dark matter first order phase transition may leave a trace on the observed CMB radiation. The ISW effect would induce that trace. The ISW effect, is the gravitational redshift of the last-scattering photons due to gravitational field gradients along their trajectory. 
Explicitly, the contribution of the ISW to the temperature fluctuations of the CMB is given by:
\begin{equation} \label{eq:1}
\frac{\delta T}{T} = \int_{\eta_r}^{\eta_0} (\phi' - \psi') d\eta
\end{equation}

Where $\frac{\delta T}{T}$ is the temperature fluctuation of the CMB from a specific direction, $\phi$ and $\psi$ are gravitational potentials (See next chapters for conventions), tags indicate derivative by conformal time. The integral is taken along the Photon's trajectory, $\eta$ is used as its parameter and it is assumed the Photon was emitted at $\eta_r$ and measured at $\eta_0$.

Under the assumption of linear evolution, the PT's direct contribution can be computed by plugging the gravitational potential perturbations induced by the PT to Eq. \ref{eq:1}. This note aims to demonstrate how this is done. 

Schematically, this computation contains 3 major steps:
\begin{enumerate}
    \item Simulating the PT
    \item Computing the corresponding stress-energy tensor
    \item Computing the metric fluctuations sourced by the stress-energy tensor
\end{enumerate}
The following chapters will expand on each step. 


\section{Simulating the PT}
Since this work aims to probe the effect of a possible first-order phase transition (FOPT), a basic model is a good first step, and perhaps is entirely sufficient. The basic case considered in this work is of a single self-interacting scalar field, whose Lagrangian is given by: 

\begin{equation} \label{scalar_field_lagrangian}
    \mathcal{L}(\varphi) = \frac{1}{2}\partial_\mu\varphi\partial^\mu\varphi - V(\varphi)
\end{equation}

And the potential:

\begin{equation} \label{FOPT_potential}
    V(\varphi) = \frac{\lambda}{8}(\varphi ^2 - \varphi_0 ^2) ^ 2 - \epsilon\lambda\varphi_0^3(\varphi + \varphi_0)
\end{equation}



Analysis of an FOPT emerging from this model is performed in \cite{coleman}. This analysis yields important conclusions:
\begin{enumerate}
    \item The bubble has an $O(4)$ symmetry (in imaginary time).
    \item  The bubble wall speed of expansion reaches the speed of light in a time corresponding to its small initial radius.
    \item The released energy resides in the bubble's wall, and as a result of \item[1], is split evenly between kinetic and gradient energy.
\end{enumerate}
These properties yield a concrete picture of a single bubble's evolution. For a full treatment of an FOPT though, an accompanying picture of the bubble collisions is important. A fundamental approach would use the Klein Gordon's equations for the bubble's propagation. Instead, the approach carried out in this work takes after \cite{kosowsky}, and the envelope approximation is used. 

The collision of 2 vacuum bubbles creates a region in which the field's value oscillates rapidly. In the envelope approximation the oscillations within this region are neglected.

\section{Stress-Energy Tensor}
Our goal is to compute the components of the stress energy tensor, $T_{\mu\nu}$
as a function of time during the FoPT. For that, we follow and expand on the 
same computation as in \cite{kosowsky_envelope_approximation}.
First, we derive the relations between the energy density, pressure and anisotropic stress
that are sourced by the scalar field. Then, we write an approximate formula for their computation in momentum space
which may be implemented numerically.

\subsection{Components}
The stress energy tensor of a scalar field with the Lagrangian in Eq. \ref{scalar_field_lagrangian}
is given by:

\begin{equation} \label{Stress-Energy-Tensor}
    T_{\mu\nu} = \partial_\mu \varphi \partial_\nu \varphi - g_{\mu\nu}\mathcal{L}(\varphi)
\end{equation}

From it, we extract the energy density, $\rho$, pressure, $P$ and anisotropic stress tensor, $\Pi^i_j$:

\begin{equation} \label{energy density}
    \rho = T^0_0 = \frac{1}{2}(\partial_t\varphi)^2 + \frac{1}{2}\partial_i\varphi\partial_i\varphi + V(\varphi)
\end{equation}
\begin{equation} \label{pressure}
    P = T^i_i = \frac{1}{2}(\partial_t \varphi) ^2 - \frac{1}{6}\partial_i\varphi\partial_i\varphi - V(\varphi)
\end{equation}
\begin{equation}\label{anisotropic stress}
    \Pi^i_j=-T^i_j-P\delta^i_j=\partial_i\varphi \partial_j\varphi-\frac{1}{3}\delta^i_j\partial_k\varphi\partial_k\varphi
\end{equation}

Since we are interested in the scalar fluctuations, we extract the scalar
part of the anisotropic stress tensor, which is given by:

\begin{equation} \label{scalar part of anisotropic stress tensor}
    {\Pi^{(s)}}^i_j = (\partial^i\partial_j-\frac{1}{3}\delta^i_j\nabla^2)\mathcal{N}_\varphi
\end{equation}

where $\mathcal{N}_\varphi$ is the scalar anisotropic stress potential.
Up to a function $\lambda$ s.t $\nabla^4\lambda = 0$, Eq. \ref{scalar part of anisotropic stress tensor} can be inverted to give:
\begin{equation}
    \mathcal{N}_\varphi =\frac{1}{\nabla^4}(\frac{3}{2}\partial^j\partial_i\Pi^i_j)
\end{equation}

\subsection{Momentum Space Formula}
As a step towards numerical computation, we consider the 
stress energy tensor in momentum space (i.e the fourier transform) which we denote by a $\sim$.
As is apparent from equations \ref{energy density}, \ref{pressure} and \ref{anisotropic stress} (and will be clarified below)
the building blocks for this computation are integrals of the form:

\begin{equation}\label{Basic Momentum Space integral}
    \widetilde{\partial_i\varphi\partial_j\varphi}(\overrightarrow{k}, t) = \int_{R^3}\partial_i\varphi\partial_j\varphi e^{-i\overrightarrow{k}\cdot\overrightarrow{x}}d^3x
\end{equation}

\begin{equation}\label{Basic Volume Momentum Space integral}
    \widetilde{V}(\varphi)(\overrightarrow{k}, t) = \int_{R^3}V(\varphi)(\overrightarrow{x}, t) e^{-i\overrightarrow{k}\cdot\overrightarrow{x}}d^3x
\end{equation}
Where $\overrightarrow{k}$ is the momentum.

We start with Eq. \ref{Basic Momentum Space integral}. By decomposing it to a sum over single bubble contributions, we get:

\begin{equation} \label{decomposition of bubbles}
    \int_{R^3} d^3x=\sum_n \int_{V_n}d^3x
\end{equation}
Where the index $n$ runs over all bubbles and $V_n$ is the region of the $n$th bubble.

As is demonstrated in \cite{coleman}, a single bubble's field configuration obeys $O(4)$ symmetry in imaginary time. Thus, we assume that within the region $V_n$ 
one can write:
$\varphi(\overrightarrow{x}, t) = \varphi(r, t)$
where $r$ is the radial parameter around the bubble's nucleation site.
As a consequence, the expression for the scalar field derivative simplifies to:

\begin{equation} \label{spherical symmetri}
    \partial_i\varphi = \frac{dr}{dx^i} \frac{d\varphi}{dr} = \hat{x_i} \frac{d\varphi}{dr}
\end{equation}

Plugging Eqs. \ref{decomposition of bubbles} and \ref{spherical symmetri} into Eq. \ref{Basic Momentum Space integral} yields:

\begin{equation} \label{intermediate momentum space integral}
    \begin{aligned}
        \widetilde{\partial_i\varphi\partial_j\varphi}(\overrightarrow{k}, t) &= \sum_n \int_{V_n}\hat{{x}}^{(n)}_i\hat{x}^{(n)}_j{(\frac{d\varphi}{dr^{(n)}})}^2 e^{-i\overrightarrow{k}\cdot\overrightarrow{x}}d^3x\\
        & \approx \sum_n e^{-i\overrightarrow{k}\cdot\overrightarrow{x}^{(n)}} \int_{S^{(n)}(t)} \hat{x}_i^{(n)}\hat{x}_j^{(n)}\int_0^{R^{(n)}(t)}{(\frac{d\varphi}{dr^{(n)}})}^2 e^{-ikr^{(n)}cos(\theta^{(n)})} (r^{(n)})^2 dr^{(n)}d^2\Omega^{(n)}\\
        & \approx \sum_n e^{-i\overrightarrow{k}\cdot\overrightarrow{x}^{(n)}} \int_{S^{(n)}(t)} \hat{x}_i^{(n)}\hat{x}_j^{(n)} e^{-ikR^{(n)} cos(\theta^{(n)})} \int_0^{R^{(n)}(t)}{(\frac{d\varphi}{dr^{(n)}})}^2 (r^{(n)})^2dr^{(n)}d^2\Omega^{(n)}
    \end{aligned}   
\end{equation}

Where a super script of $(n)$ denotes bubble dependence, $k = |\overrightarrow{k}|$, $R$ is the bubble radius, $r$ is the
radial coordinate, $\theta$ is the axial angle, $\overrightarrow{x}^{(n)}$ is the vector pointing to the bubble's center, and $d^2\Omega$ is a unit sphere surface element. In the last line, we used the approximation that the bubble width is small with respect to the bubble radius (i.e $\frac{d\varphi}{dr}$ is nonzero only at $r=R(t)$). 
In the second equality, we limited the region of integration to $S^{(n)}(t)$, the region on the bubble's surface that remains uncollided at time $t$. In doing so, we neglected the contribution of the bubble collisions to the stress energy tensor.

An important simplification to Eq. \ref{intermediate momentum space integral} is achieved by noting the relation between the 
integral over the field's squared derivative, and the vacuum energy liberated by the bubble expansion.
As demonstrated in \cite{kosowsky_envelope_approximation}, for $t \gg R_0$ where $R_0$ is the
bubble's radius at nucleation, the bubble's energy is distributed evenly between its potential and kinetic energy.
Thus, we have:
\begin{equation}
    \begin{aligned}
        & E_p(t) = 4\pi \int_0^{R(t)}\frac{1}{2} {(\frac{d\varphi}{dr})}^2r^2dr\\
        & E_k(t) = 4\pi \int_0^{R(t)}\frac{1}{2}{(\frac{d\varphi}{dt})}^2 r^2 dr\\
        & E_p(t)\approx E_k(t) \approx \frac{1}{2}E(t)=\frac{1}{2}\cdot \frac{4\pi}{3} R^3(t) \cdot \rho_{vac}
    \end{aligned}
\end{equation}
Upon insertion to Eq. \ref{intermediate momentum space integral}, we get:
\begin{equation} \label{Simplified momentum space surface integral}
    \widetilde{\partial_i\varphi\partial_j\varphi}(\overrightarrow{k}, t) \approx \sum_n \frac{E^{(n)}(t)e^{-i\overrightarrow{k}\cdot\overrightarrow{x}^{(n)}}}{4\pi}\int_{S^{(n)}(t)} \hat{x}_i^{(n)}\hat{x}_j^{(n)} e^{-ikR^{(n)} \mu^{(n)}} d^2 \Omega^{(n)}
\end{equation}
Eq. \ref{Simplified momentum space surface integral} may be readily computed using numerical methods on a simulated bubble configuration.
In conclusion, for intergrals of the form of Eq. \ref{Basic Momentum Space integral}, Eq. \ref{Simplified momentum space surface integral} is used, whereas
Eq. \ref{Basic Volume Momentum Space integral} may be computed directly using numerical methods.

\section{Metric Fluctuations}

\appendix
\section{Conventions}

%beginning of Diagram code
%\begin{tikzpicture}[node distance=2cm] \label{Strategy Diagram}
%\node (first step) [startstop] {PT Simulation}
%\node (second step) [startstop, below of=first step] {$T^{\mu\nu}$}
%\node (third step) [start]
%\draw [arrow] (start) -- (in1);
%\end{tikzpicture}
\bibliographystyle{plain}
\bibliography{bib}
\end{document}
